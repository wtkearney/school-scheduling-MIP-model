% !TEX TS-program = pdflatex
% !TEX encoding = UTF-8 Unicode

\documentclass[12pt]{article}

\usepackage[utf8]{inputenc} % set input encoding (not needed with XeLaTeX)

%%% PACKAGES
\usepackage{amsmath}
\usepackage{amsfonts}
\usepackage{graphicx}
\usepackage{booktabs}	 % for much better looking tables
\usepackage{array}	 % for better arrays (eg matrices) in maths
\usepackage{breqn} 	% breaks lines
\usepackage{authblk}	% helps with authors on title page
\usepackage{multirow}

%----------------------------------------------------------------------------
\begin{document}
%----------------------------------------------------------------------------
\renewcommand\Authfont{\small}
\renewcommand\Affilfont{\itshape\footnotesize}
%----------------------------------------------------------------------------
\title{Middle School Course Scheduling as a Linear Programming Model}
%----------------------------------------------------------------------------
\author[1]{Will Kearney}
\author[1]{Sonimar Poppe}
\author[2]{Niguel Morfin}
\author[3]{Asrar Ahmed Syed}
\affil[1]{Portland Public Schools}
%----------------------------------------------------------------------------
%\date{} % Activate to display a given date or no date (if empty),
         % otherwise the current date is printed 
%----------------------------------------------------------------------------
\maketitle
%----------------------------------------------------------------------------
\begin{abstract}
This is something we need to do eventually.
\end{abstract}
%----------------------------------------------------------------------------
\section{Model Formulation}


\begin{table}[]
\centering
\caption{The data notation used in our formulation}
\label{tab:data-notation}
\begin{tabular}{ll}
\hline
$maxClassSize_{c}$ & The maximum number of students \\
 & permitted in class $c$ \\
$numRequiredCore$ & The number of required core classes per student \\
\hline
\end{tabular}
\end{table}


\subsection{Decision Variables}

% student decision variables
\begin{equation}
	X_{s,c,p} = 
	\begin{cases}
		1, & \text{if student}\ s \text{ is assigned to course}\ c \text{ in period}\ p	\\
		0, & \text{otherwise}
	\end{cases}
\end{equation}

and

% staff decision variables
\begin{equation}
	Y_{t,c,p} = 
	\begin{cases}
		1, & \text{if teacher}\ t \text{ is assigned to course}\ c \text{ in period}\ p	\\
		0, & \text{otherwise}
	\end{cases}
\end{equation}

\subsection{Constraints}

The first constraint set deals primarily with capacity.

\begin{equation} \label{eq:every-student-fully-scheduled}
	\displaystyle \sum_{c} X_{s,c,p} = 1 \quad \forall s,p
\end{equation}

\begin{equation} \label{eq:max-one-teacher-per-course-period}
	\displaystyle \sum_{t} Y_{t,c,p} = 1 \quad \forall c,p
\end{equation}

Constrainst~\ref{eq:every-student-fully-scheduled} ensures that every student is full scheduled (i.e. taking exactly one course every period of the day). Similarly, constrainst~\ref{eq:max-one-teacher-per-course-period} ensures that at maximum only one teacher can be assigned to a given course and period.

\begin{equation} \label{eq:only-one-class}
	\displaystyle \sum_{p} X_{s,c,p} \leq 1 \quad \forall s,c
\end{equation}

Constraint~\ref{eq:only-one-class} dictates that a student can't take a given class more than once per day.

\begin{equation} \label{eq:number-courses-taught}
\begin{split}
	\displaystyle \sum_{c,p} Y_{t,c,p} \geq 1 \quad \forall t \\
	\displaystyle \sum_{c,p} Y_{t,c,p} \leq 5 \quad \forall t
\end{split}
\end{equation}

Constraint~\ref{eq:number-courses-taught} is designed to limit the number of courses assigned to each teacher between 1 and 5.

\begin{equation} \label{eq:class-size-capacity}
	\displaystyle \sum_{s} X_{s,c,p} \leq maxClassSize_{c} \quad \forall c,p
\end{equation}

Constraint~\ref{eq:class-size-capacity} restricts the maximum number of students assigned to a class and period to the maximum number of seats available (or some arbitrary upper bound).


\subsection{Specific Course Requirement}

\begin{equation} \label{eq:num-required-core}
	\displaystyle \sum_{c,p} X_{s,c,p} \cdot core_{c} = numRequiredCore \quad \forall s
\end{equation}

Constraint~\ref{eq:num-required-core} ensures that every student is taking a set number of core classes per day. In our formulation, for an 8 period day $numRequiredCore$ is equal to 5.

\begin{equation} \label{eq:lunch-constraint}
	X_{s,lunch,4} + X_{s,lunch,5} = 1 \quad \forall s
\end{equation}

Constraint~\ref{eq:lunch-constraint} dictates that each student is assigned to lunch during either period 4 or 5.


\end{document}












