% !TEX TS-program = pdflatex
% !TEX encoding = UTF-8 Unicode

\documentclass[12pt]{article}

\usepackage[utf8]{inputenc} % set input encoding (not needed with XeLaTeX)

%%% PACKAGES
\usepackage{amsmath}
\usepackage{amsfonts}
\usepackage{graphicx}
\usepackage{booktabs} % for much better looking tables
\usepackage{array} % for better arrays (eg matrices) in maths



\title{MIP School Scheduling}
\author{Will Kearney, Sonimar Poppe, Niguel Morfin, and Asrar Ahmed Syed}
%\date{} % Activate to display a given date or no date (if empty),
         % otherwise the current date is printed 

\begin{document}
\maketitle

\section{Model Formulation}

\begin{table}[]
\centering
\caption{The notation used in our formulation}
\label{notationl}
\begin{tabular}{ll}
\hline
S & Set of students \\
T & Set of teachers \\
C & Set of courses  \\
P & Set of periods  \\
D & Set of days    \\ \hline
\end{tabular}
\end{table}

\subsection{Variables}

% decision variable
\begin{equation}
	X_{s,t,c,p,d} \in \{0,1\}
\end{equation}

where $X_{s,t,c,p,d}$ is a binary decision variable indicating whether student $s \in S$ is assigned to teacher $t \in T$ for class $c \in C$ in period $p \in P$ on day $d \in D$ ($X_{s,t,c,p,d}=1$ if student is assigned, 0 otherwise).


\subsection{Constraints}

% Ensures every student is fully scheduled
\begin{equation} \label{eq:students_fully_scheduled}
	\displaystyle\sum_{t=1}\sum_{c=1}\sum_{p=1} X_{s,t,c,p,d}=7 \quad \forall s,d
\end{equation}

Constraint~\ref{eq:students_fully_scheduled} ensures that every student is fully scheduled (i.e. taking a full 7 periods each day).



\end{document}

