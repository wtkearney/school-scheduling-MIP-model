% !TEX TS-program = pdflatex
% !TEX encoding = UTF-8 Unicode

\documentclass[12pt]{article}

\usepackage[utf8]{inputenc} % set input encoding (not needed with XeLaTeX)

%%% PACKAGES
\usepackage{amsmath}
\usepackage{amsfonts}
\usepackage{graphicx}
\usepackage{booktabs} % for much better looking tables
\usepackage{array} % for better arrays (eg matrices) in maths
\usepackage{breqn} % breaks lines



\title{MIP School Scheduling}
\author{Will Kearney, Sonimar Poppe, Niguel Morfin, and Asrar Ahmed Syed}
%\date{} % Activate to display a given date or no date (if empty),
         % otherwise the current date is printed 

\begin{document}
\maketitle

\section{Model Formulation}

\begin{table}[]
\centering
\caption{The notation used in our formulation}
\label{notationl}
\begin{tabular}{ll}
\hline
S & Set of students \\
T & Set of teachers \\
C & Set of courses  \\
P & Set of periods  \\
D & Set of days    \\ \hline
\end{tabular}
\end{table}

\subsection{Variables}

% decision variable
\begin{equation}
	X_{s,t,c,p,d} \in \{0,1\}
\end{equation}

where $X_{s,t,c,p,d}$ is a binary decision variable indicating whether student $s \in S$ is assigned to teacher $t \in T$ for course $c \in C$ in period $p \in P$ on day $d \in D$ ($X_{s,t,c,p,d}=1$ if student is assigned, 0 otherwise).


\begin{equation}
	num\_students\_assigned_{t,c,p,d} \in \mathbb{Z}
\end{equation}

where $num\_students\_assigned_{t,c,p,d}$ is an integer variable counting the number of students assigned to a given teacher $t \in T$ for course $c \in C$ in period $p \in P$ on day $d \in D$.


\begin{equation}
	teacher\_scheduled\_indicator_{t,c,p,d} \in \{0,1\}
\end{equation}

where $teacher\_scheduled\_indicator_{t,c,p,d}$ is a binary variable indicating if a given teacher $t \in T$ is scheduled for course $c \in C$ in period $p \in P$ on day $d \in D$.


\subsection{Constraints}

% Ensures every student is fully scheduled
\begin{equation} \label{eq:students_fully_scheduled}
	\displaystyle\sum_{t=1}\sum_{c=1}\sum_{p=1} X_{s,t,c,p,d}=7 \quad \forall s,d
\end{equation}

Constraint~\ref{eq:students_fully_scheduled} ensures that every student is fully scheduled (i.e. taking a full 7 periods each day).

% Counts the number of students assigned
\begin{equation} \label{eq:count_num_students_assigned}
	num\_students\_assigned_{t,c,p,d} = \displaystyle\sum_{s=1} X_{s,t,c,p,d}  \quad \forall t,c,p,d
\end{equation}

Constraint~\ref{eq:count_num_students_assigned} links the $num\_students\_assigned_{t,c,p,d}$ variable by counting the number of students assigned for each teacher $t \in T$, course $c \in C$, period $p \in P$, and day $d \in D$.

% Link teacher scheduled indicator variable
\begin{multline}\label{eq:link_teacher_scheduled_indicator}
%\begin{equation} \label{eq:link_teacher_scheduled_indicator}
	teacher\_scheduled\_indicator_{t,c,p,d} \cdot M \\ \geq num\_students\_assigned_{t,c,p,d}  \quad \forall t,c,p,d
%\end{equation}
\end{multline}

Constraint~\ref{eq:link_teacher_scheduled_indicator} sets $teacher\_scheduled\_indicator_{t,c,p,d}=1$ if $num\_students\_assigned_{t,c,p,d}>0$ through the use of a sufficiently large `Big M' variable $M$.

% Every teacher only teaches one course for each period on each day
\begin{equation} \label{eq:teachers_not_double_scheduled}
	\displaystyle\sum_{c=1} teacher\_scheduled\_indicator_{t,c,p,d} \leq 1 \quad \forall t,p,d
\end{equation}

Constraint~\ref{eq:teachers_not_double_scheduled} ensures every teacher is teaching no more than one course during each period on every day.

% Force student assignment if currently enrolled in core class
\begin{equation} \label{eq:force_current_assignments}
	\displaystyle\sum_{t=1}\sum_{p=1}\sum_{d=1} X_{s,t,c,p,d} = 1 \quad \forall s,c \in S,C | P(s,c)
\end{equation}

where $P(s,c)$ is the property in which student $s$ is currently enrolled in course $c$ and course $c$ is a `core course'. Constraint~\ref{eq:force_current_assignments} requires a student to be enrolled in a course if it is designated as a `core course' and they are currently enrolled in it. Note that the course does not need to be in the same period on the same day, or taught by the same teacher.

% Lunch
\begin{equation} \label{eq:assign_lunch}
	 X_{s,t,c_{lunch},p_{4},d} +  X_{s,t,c_{lunch},p_{5},d} = 1 \quad \forall s,t,d
\end{equation}

Constraint~\ref{eq:assign_lunch} ensures every student and every teach is assigned one lunch period during either 4th or 5th period.

\end{document}












